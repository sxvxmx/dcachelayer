\usepackage[12pt]{extsizes}
\usepackage[T2A]{fontenc}
\usepackage[utf8]{inputenc}
\usepackage[english, russian]{babel}
\usepackage[left=3cm,right=1.5cm,top=2cm,bottom=2cm]{geometry}
\usepackage{hyperref}
\usepackage{amssymb,amsmath,amsthm}
\usepackage{cleveref}
\usepackage{caption}
\usepackage{mathtools}


\usepackage{graphicx}

%\полуторный интервал
\usepackage{setspace}
\onehalfspacing

%простое дерево
\usepackage{dirtree}

%формулы
\usepackage{mathtools}
% \everymath{\displaystyle}

%красные строки (не рекомендуется)
% \parindent = 1,25cm
% \usepackage{indentfirst}



%some commands for better experience
\theoremstyle{definition}
\newtheorem{define}{Определение}
\newcommand*{\defi}[1]{\begin{define} #1 \end{define}}

\theoremstyle{plain}
\newtheorem{theorem}{Теорема}
\newcommand*{\thrm}[1]{\begin{theorem} #1 \end{theorem}}

\theoremstyle{plain}
\newtheorem{lemma}{Лемма}
\newcommand*{\lemm}[1]{\begin{lemma} #1 \end{lemma}}

\newcommand*{\prf}[1]{\begin{proof} #1 \end{proof}}

\theoremstyle{plain}
\newtheorem{example}{Пример}
\newcommand*{\exmp}[1]{\begin{example} #1 \end{example}}

\newcommand*{\R}{\mathbb{R}}
\newcommand*{\und}[2]{\underset{#1}{#2}}
